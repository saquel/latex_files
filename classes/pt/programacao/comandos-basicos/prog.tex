\documentclass[12pt]{article}
\usepackage[T1]{fontenc}
\usepackage[utf8]{inputenc}  % só se necessário
\usepackage[brazil]{babel}
\usepackage{textcomp}
\usepackage{amsmath}
\usepackage{geometry}
\usepackage{xcolor}
\usepackage{listings}
\usepackage{titlesec}
\usepackage{fancyhdr}
\usepackage{hyperref}

\geometry{a4paper, margin=2.5cm}

\definecolor{codegray}{gray}{0.9}
\definecolor{keywordcolor}{rgb}{0.0,0.0,0.7}
\definecolor{stringcolor}{rgb}{0.2,0.6,0.2}

\lstdefinestyle{pythonstyle}{
	backgroundcolor=\color{codegray},
	language=Python,
	basicstyle=\ttfamily\small,
	keywordstyle=\color{keywordcolor}\bfseries,
	stringstyle=\color{stringcolor},
	commentstyle=\color{gray},
	showstringspaces=false,
	numbers=left,
	numberstyle=\tiny,
	frame=single,
	breaklines=true
}

\titleformat{\section}{\normalfont\Large\bfseries}{\thesection.}{1em}{}
\pagestyle{fancy}
\fancyhf{}
\rhead{Python Básico}
\lhead{Introdução}
\rfoot{\thepage}

\title{Introdução à Programação com Python}
\author{Professor Isaque}
\date{}

\begin{document}
	
	\maketitle
	
	\section{O que é uma variável?}
	Uma \textbf{variável} é um nome que damos a um espaço na memória do computador onde podemos guardar informações. Por exemplo:
	
	\begin{lstlisting}[style=pythonstyle]
		nome = "Isaque"
		idade = 13
	\end{lstlisting}
	
	\section{Tipos de variáveis}
	\begin{itemize}
		\item \textbf{int}: números inteiros (ex: 10, -5)
		\item \textbf{float}: números com vírgula (ex: 1.75)
		\item \textbf{str}: texto (ex: "Olá")
		\item \textbf{bool}: verdadeiro ou falso (True / False)
	\end{itemize}
	
	\begin{lstlisting}[style=pythonstyle]
		idade = 10
		altura = 1.75
		nome = "Joao"
		ligado = True
	\end{lstlisting}
	
	\section{Listas, Tuplas e Dicionários}
	
	\subsection*{Lista (list)}
	Coleção de itens que pode mudar:
	\begin{lstlisting}[style=pythonstyle]
		frutas = ["maca", "banana", "uva"]
	\end{lstlisting}
	
	\subsection*{Tupla (tuple)}
	Coleção imutável:
	\begin{lstlisting}[style=pythonstyle]
		cores = ("azul", "vermelho", "verde")
	\end{lstlisting}
	
	\subsection*{Dicionário (dict)}
	Chaves com valores:
	\begin{lstlisting}[style=pythonstyle]
		aluno = {"nome": "Ana", "idade": 14}
	\end{lstlisting}
	
	\section{Comando \texttt{print()}}
	Usado para exibir algo na tela:
	
	\begin{lstlisting}[style=pythonstyle]
		print("Ola, mundo!")
		print("A soma e", 3 + 5)
	\end{lstlisting}
	
	\section{F-strings e \texttt{.format()}}
	Permite incluir variáveis dentro de textos:
	
	\begin{lstlisting}[style=pythonstyle]
		idade = 14
		print(f"Voce tem {idade} anos")
		
		nota = 9.5
		print("Sua nota foi {:.1f}".format(nota))
	\end{lstlisting}
	
	\section{Comando \texttt{input()}}
	Lê uma informação do usuário:
	\begin{lstlisting}[style=pythonstyle]
		nome = input("Qual e o seu nome? ")
		print(f"Ola, {nome}!")
	\end{lstlisting}
	
	\section{Conversões \texttt{int()} e \texttt{float()}}
	Convertem texto em número:
	\begin{lstlisting}[style=pythonstyle]
		idade = int(input("Digite sua idade: "))
		altura = float(input("Digite sua altura: "))
	\end{lstlisting}
	
	\section{Condicional \texttt{if/else}}
	
	Serve para fazer escolhas:
	
	\begin{lstlisting}[style=pythonstyle]
		if idade >= 18:
		print("Maior de idade")
		else:
		print("Menor de idade")
	\end{lstlisting}
	
	\section{Repetição com \texttt{while}}
	
	Repete algo enquanto a condição for verdadeira:
	
	\begin{lstlisting}[style=pythonstyle]
		numero = 1
		while numero <= 5:
		print(numero)
		numero += 1
	\end{lstlisting}
	
	\section{Repetição com \texttt{for}}
	
	Percorre uma sequência:
	
	\begin{lstlisting}[style=pythonstyle]
		for i in range(1, 6):
		print(i)
	\end{lstlisting}
	
	\section{Resumo dos Comandos}
	
	\begin{tabular}{|l|l|}
		\hline
		\textbf{Comando} & \textbf{Função} \\
		\hline
		\texttt{print()} & Exibir na tela \\
		\texttt{input()} & Entrada do usuário \\
		\texttt{int()}   & Converter para inteiro \\
		\texttt{float()} & Converter para decimal \\
		\texttt{if/else} & Decisão \\
		\texttt{while}   & Repetição condicional \\
		\texttt{for}     & Repetição com sequência \\
		\texttt{list[]}  & Lista (mutável) \\
		\texttt{tuple()} & Tupla (imutável) \\
		\texttt{dict\{\}}& Dicionário \\
		\hline
	\end{tabular}
	
\end{document}
