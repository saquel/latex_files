\section{Lista 01 - Intergraus Matemática}
\url{https://www.qconcursos.com/questoes-do-enem/questoes?difficulty%5B%5D=2&discipline_ids%5B%5D=13&subject_ids%5B%5D=20321}

\begin{enumerate}
	
\begin{table}[h]
	\centering
	\renewcommand{\arraystretch}{1.3} % Espaçamento entre linhas da tabela
	\begin{tabular}{ |c|c|c| }
		\hline
		Código & Matéria & Conteúdo \\ 
		\hline
		Q2543137 & Matemática & Aritmética e Problemas, Porcentagem \\  
		\hline
	\end{tabular}
\end{table}

\begin{table}[h]
	\centering
	\renewcommand{\arraystretch}{1.3}
	\begin{tabular}{ |c|c|c|c| }
		\hline
		Ano & Banca & Órgão & Prova \\  
		\hline
		2023 & INEP & ENEM & \resizebox{10cm}{!}{INEP - 2023 - ENEM - Exame Nacional do Ensino Médio - Primeiro e Segundo - PPL (2° Aplicação)} \\
		\hline
	\end{tabular}
\end{table}
	\item Uma escola realizou uma pesquisa entre todos os seus estudantes e constatou que três em cada dez deles estão matriculados em algum curso extracurricular de língua estrangeira. \\
	Em relação ao número total de estudantes dessa escola, qual porcentagem representa o número de alunos matriculados em algum curso extracurricular de língua estrangeira?
	
	\begin{enumerate}
		\item 0,3%
		\item 0,33%
		\item 3%
		\item 30%
		\item 33%
	\end{enumerate}
 
	\newpage
	\TabelaQuestao{Q2543136}{Matemática}{Aritmética e Problemas ,
		Porcentagem} \\
	
	\TabelaProva{2023}{INEP}{ENEM}{INEP - 2023 - ENEM - Exame Nacional do Ensino Médio - Primeiro e Segundo - PPL ( 2° Aplicação)}
	
	\item Toda a iluminação de um escritório é feita utilizando-se 40 lâmpadas incandescentes que produzem 600 lúmens (lúmen = unidade de energia luminosa) cada. O gerente planeja reestruturar o sistema de iluminação desse escritório, utilizando somente lâmpadas fluorescentes que produzem 1 600 lúmens, para aumentar a quantidade de energia luminosa em 50\%. Para alcançar seu objetivo, a quantidade mínima de lâmpadas fluorescentes que o gerente desse escritório deverá instalar é
	
	\begin{enumerate}
		\item 10.
		\item 14.
		\item 15.
		\item 16.
		\item 23.		
	\end{enumerate}
	
		\newpage
	\TabelaQuestao{Q3160688}{Matemática}{Aritmética e Problemas ,
		Regra de Três} \\
	
	\TabelaProva{2024}{INEP}{ENEM}{INEP - 2024 - ENEM - Exame Nacional do Ensino Médio - PPL - primeiro e segundo dia}
	
	\item Um engenheiro civil organizou duas equipes de pedreiros, I e II, para a construção de um muro bastante extenso. Todos os pedreiros, de ambas as equipes, apresentam o mesmo rendimento por hora trabalhada frente à construção planejada. A equipe I era composta por 3 pedreiros, que construíram 36 metros quadrados de muro, em 2 dias, trabalhando 6 horas por dia. A equipe II era composta por 5 pedreiros, que trabalharam 8 horas diárias, durante 6 dias. 	Quantos metros quadrados a equipe II construiu a mais do que a equipe I?
	
	\begin{enumerate}
		\item 144
		\item 180
		\item 204
		\item 240
		\item 276
	\end{enumerate}

\end{enumerate}


