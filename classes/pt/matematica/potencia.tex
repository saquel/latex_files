\documentclass[a4paper,12pt]{article}
\usepackage[brazil]{babel}
\usepackage{amsmath, amssymb}
\usepackage{geometry}
\usepackage{multicol}
\geometry{left=2cm,right=2cm,top=2cm,bottom=2cm}

\begin{document}
	
	\title{Lista de Exercícios - Potenciação}
	\author{Matemática - 8º Ano}
	\date{\today}
	\maketitle
	
	\begin{multicols}{2}
		
		\section*{Definição de Potenciação}
		Resolva as expressões abaixo:
		\begin{multicols}{3}
		\begin{enumerate}
			\item $2^3$ \item  $2.5^2$ \item $(1.2)^4$
			\item $5^4$ \item $0.4^3$ \item $\left(\dfrac{2}{3}\right)^2$
			\item $7^2$ \item $1.1^5$ \item $\left(\dfrac{5}{8}\right)^3$
			\item $10^3$ \item $0.25^2$ \item $\left(\dfrac{4}{9}\right)^4$
			\item $3^5$  \item $\left(\dfrac{7}{5}\right)^2$ \item $0.333\ldots^3$ 
		\end{enumerate}
		\end{multicols}
	
		\section*{Propriedades da Potenciação}
		Resolva aplicando as propriedades da potenciação:
		\begin{multicols}{2}
		\begin{enumerate}
			\item $2^3 \times 2^4$
			\item $5^6 \div 5^2$
			\item $(3^2)^3$
			\item $7^5 \times 7^{-3}$
			\item $(2 \times 3)^4$
			\item $ \Bigg[ \bigg( \dfrac{3}{2} \bigg)^{2} \Bigg] ^{3}$
		\end{enumerate}
		\end{multicols}
		
		\section*{Potências de Base 10}
		Resolva as seguintes expressões:
		\begin{enumerate}
			\item $10^3$;\quad $10^6$;\quad $10^{-2}$
			\item $5 \times 10^4$;\quad $3,2 \times 10^{-3}$
			\item $4,25 \times 10^{-5}$;\quad $23,06 \times 10^{-3}$
		\end{enumerate}
		
		\section*{Potências de Expoente Negativo}
		Reescreva as expressões com expoente positivo:
		\begin{multicols}{2}
		\begin{enumerate}
			\item $2^{-3}$
			\item $5^{-2}$
			\item $10^{-4}$
			\item $\dfrac{1}{3^{-2}}$
			\item $\left(\dfrac{2}{5}\right)^{-3}$
			\item $ \Bigg[ \bigg( \dfrac{3}{2} \bigg)^{-2} \Bigg] ^{-3}$
		\end{enumerate}
		\end{multicols}
		
		\section*{Substituição de Variáveis}
		Considere $x = 2^3$, $y = 5^{-2}$ e $z = 10^1$. Substitua e resolva:
		\begin{enumerate}
			\item $x + y - z$
			\item $\dfrac{x^2}{z} + y$
			\item $(x \times y)^z$
		\end{enumerate}
		
		\section*{Substituição de Variáveis}
		\begin{enumerate}
			\item Sendo $a=2^{7} \times 3^{8} \times 7$ e $b= 2^{5} \times 3^{6}$, o quociente de $a$ por $b$ é igual a:	
			\begin{multicols}{2}	
				\begin{enumerate}				
					\item 252   \item 126
					\item 36 	\item 48
				\end{enumerate}
			\end{multicols}
			\item Um número é expresso por $(2^{6} \div 2^{4}) + 2^{2}$
			\begin{multicols}{2}	
				\begin{enumerate}				
					\item $2^{3}$   \item $2^{0}$
					\item $2^{4}$ 	\item $2^{5}$
				\end{enumerate}
			\end{multicols}
			\item Se $x=3^{6}$ e $y=9^{3}$, podemos afirmar que:
			
				\begin{enumerate}				
					\item $x$ é o dobro de y   \item $x=y$
					\item $x-y=1$	\item $y$ é o triplo de x
				\end{enumerate}
		
		\end{enumerate}
		
		
		\section*{Expressões Numéricas com Potências}
		Calcule o valor das expressões abaixo:
		\begin{enumerate}
			\item $3^3 + 4^2 - 2^4$
			\item $5^2 - 3^3 + 10^1$
			\item $\left(2^3 + 3^2\right) \times 5^{-1}$
			\item $\dfrac{7^2 - 2^3}{3^2}$
			\item $\left(\dfrac{4^3}{2^2}\right)^2$
			\item $1.2^3 + \left(\dfrac{5}{6}\right)^2 - 0.4^4$
			\item $\dfrac{2.5^2 - 1.1^3}{0.5^2} + \left(\dfrac{3}{4}\right)^3$
			\item $ [ (0,4)^{2} ]^{10} \div [ (0,4)^{9} \times (0,4)^{7} \times 0,4] $
		\end{enumerate}
		
		\section*{Problemas com Potenciação}
		Resolva os problemas abaixo:
		\begin{enumerate}
			\item Um micróbio se divide em duas partes idênticas a cada minuto. Se começarmos com um único micróbio, quantos micróbios teremos após 10 minutos?
			\item Uma cidade tem uma população de $10^5$ habitantes. Se a população dobra a cada 20 anos, quantos habitantes haverá em 60 anos?
			\item Um fio elétrico tem $10^{-3}$ metros de espessura. Quantos fios são necessários para formar 1 metro de espessura?
			\item Um grão de arroz tem aproximadamente $5 \times 10^{-2}$ gramas. Quantos grãos são necessários para formar 1 quilograma?
		\end{enumerate}		
	\end{multicols}

	\hfill
	 
	5. A massa do Sol é de $1 980 000 000 000 000 000 000 000 000$ toneladas e a massa da Terra é de $5 980 000 000 000 000 000 000 000$ kg
	\begin{enumerate}
		\item[(a)] Escreva em notação científica a massa do Sol e a massa da Terra em quilos
		\item[(b)] Quantas vezes a massa do Sol é maior que a massa da Terra?
	\end{enumerate}
	
\end{document}
