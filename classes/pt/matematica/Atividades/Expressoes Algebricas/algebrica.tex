\documentclass[a4paper, landscape]{article}
\usepackage[brazil]{babel}
\usepackage{fancyhdr}  % Pacote para editar cabeçalhos
\usepackage{graphicx}   % Pacote para inserir imagens
\usepackage{amsmath, amssymb}
\usepackage{geometry}
\usepackage{multicol}
\geometry{left=1cm,right=1cm,top=1cm,bottom=1cm}
\usepackage{enumitem}	
\usepackage{float} %for figure
\usepackage{titlesec}
\usepackage{setspace} %space between lines
\usepackage[many]{tcolorbox}  % for colores boxex (tikz and xcolor included)
\titleformat{\section}{\footnotesize\bfseries}{\thesection}{1em}{}


%\singlespacing 
%\onehalfspacing %space between lines
%\doublespacing

% Remove a linha preta do cabeçalho
%\renewcommand{\headrulewidth}{0pt}  


%\pagestyle{fancy}  % Ativa o estilo de cabeçalho
%\fancyhf{} % Limpa cabeçalhos e rodapés

% Adiciona as imagens no cabeçalho, ajustando a escala conforme necessário
%\fancyhead[L]{\hspace{5cm}\raisebox{-2cm}{\includegraphics[width=2cm]{exata.jpg}}}  % Esquerda
%\fancyhead[R]{\hspace{1cm}\raisebox{-2cm}{\includegraphics[width=4cm]{pei.jpg}}} % Direita


\newtcolorbox{boxA}{
	fontupper = \bf,
	boxrule = 1.5pt,
	colframe = white  % frame color
	%rounded corners
	% arc = 5pt   % corners roundness	
}

\begin{document}	
	


	\fontsize{10}{15}\selectfont

	\vspace*{0mm}
	
	\begin{multicols}{2}
		
	\section*{Orientações}
		\begin{figure}[H]
		\begin{center}
			\includegraphics[keepaspectratio=true, scale=0.2]{pei.jpg}
		\end{center}
	\end{figure}
	
	\section*{Expressão Algébrica} 
				
	\begin{enumerate}
		\item Para determinar seu lucro nas vendas do dia, em reais, Kátia usa a fórmula matemática:  
		\boxed{  L = 8 \cdot u + 4,20  }
		Sendo $L$, o lucro das vendas e $‘u’$ o número de unidades vendidas, qual o lucro de Kátia em um dia que ela vender 45 unidades do produto?
		
		\item Júnior é minerador de ouro. Certo dia teve que alugar uma máquina, cujo valor a pagar é calculado de acordo com a expressão do quadro abaixo. \boxed{  L = 280 \cdot D + 10 \cdot Q}
		Nessa expressão, $V$ corresponde ao valor a pagar, $D$ ao número de dias e $Q$ aos quilômetros rodados. Se, Júnior, alugou por 7 dias e rodou 100 quilômetros, o valor a pagar pelo aluguel será:
		
		\item  Na expressão $x^2 + 5\cdot x - 8\cdot x \cdot y$. Calcule quando $x$ e $y$ assumir os valores: 
		\vspace{-5mm}
		\begin{multicols}{4}
			\begin{enumerate}[]
				\item $x = 2 $ \\ $y = 2 $
				\item $x = 5 $ \\ $ y = -2 $
				\item $x = -3$ \\ $ y = 4 $
				\item $x = 2^3$ \\ $ y = 3^2 $
			\end{enumerate}
		\end{multicols}
		
		
	\end{enumerate}
	
	\end{multicols}
	
		
		
		
		
		
\end{document}
