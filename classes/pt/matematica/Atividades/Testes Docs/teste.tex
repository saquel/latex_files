\documentclass[a4paper,landscape]{article}
\usepackage[brazil]{babel}
\usepackage[utf8]{inputenc}
\usepackage{graphicx}
\usepackage{amsmath,amssymb}
\usepackage{geometry}
\usepackage{enumitem}
\usepackage{titlesec}
\usepackage{setspace}
\usepackage[many]{tcolorbox}
\geometry{left=1cm,right=1cm,top=1cm,bottom=1cm}

\titleformat{\section}{\footnotesize\bfseries}{\thesection}{1em}{}

\newtcolorbox{boxA}{
	fontupper = \bf,
	boxrule = 1.5pt,
	colframe = white
}

\begin{document}
	\fontsize{10}{15}\selectfont
	
	\noindent
	\begin{minipage}[t]{0.49\textwidth}
		\begin{center}
			\includegraphics[width=3cm]{pei.jpg}
		\end{center}
		
		\section*{Expressão Algébrica}
		
		\begin{enumerate}
			\item Kátia usa a fórmula:  
			\[ L = 8 \cdot u + 4{,}20 \]
			Qual o lucro de Kátia se ela vender 45 unidades?
			
			\item Júnior aluga uma máquina com o valor:  
			\[ V = 280 \cdot D + 10 \cdot Q \]
			Se ele alugou por 7 dias e rodou 100 km, qual o valor total a pagar?
			
			\item Calcule \(x^2 + 5x - 8xy\), com:
			\begin{itemize}[label={}]
				\item \(x = 2,\ y = 2\)
				\item \(x = 5,\ y = -2\)
				\item \(x = -3,\ y = 4\)
				\item \(x = 2^3,\ y = 3^2\)
			\end{itemize}
		\end{enumerate}
	\end{minipage}	
	\hfill	
	\begin{minipage}[t]{0.49\textwidth}
		\begin{center}
			\includegraphics[width=3cm]{pei.jpg}
		\end{center}
		
		\section*{Expressão Algébrica}
		
		\begin{enumerate}
			\item Kátia usa a fórmula:  
			\[ L = 8 \cdot u + 4{,}20 \]
			Qual o lucro de Kátia se ela vender 45 unidades?
			
			\item Júnior aluga uma máquina com o valor:  
			\[ V = 280 \cdot D + 10 \cdot Q \]
			Se ele alugou por 7 dias e rodou 100 km, qual o valor total a pagar?
			
			\item Calcule \(x^2 + 5x - 8xy\), com:
			\begin{itemize}[label={}]
				\item \(x = 2,\ y = 2\)
				\item \(x = 5,\ y = -2\)
				\item \(x = -3,\ y = 4\)
				\item \(x = 2^3,\ y = 3^2\)
			\end{itemize}
		\end{enumerate}
	\end{minipage}
	
\end{document}
