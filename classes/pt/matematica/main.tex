\documentclass[12pt,a4paper]{book}

% ========================
% Pacotes
% ========================
\usepackage[utf8]{inputenc}   % Codificação
\usepackage[T1]{fontenc}      % Fontes
\usepackage[portuguese]{babel} % Idioma
\usepackage{amsmath, amssymb, amsthm} % Matemática
\usepackage{physics}          % Notação de física
\usepackage{siunitx}          % Unidades
\usepackage{graphicx}         % Inserir imagens
\usepackage{tikz}             % Diagramas
\usepackage{fancyhdr}         % Cabeçalhos personalizados
\usepackage{hyperref}         % Links no documento
\usepackage{xcolor}           % Cores
\usepackage{geometry}
\usepackage{multicol} % Texto em múltiplas colunas






% ========================
% Configurações
% ========================
\geometry{margin=2.5cm}        % Margens
\pagestyle{fancy}              % Estilo do cabeçalho/rodapé
\fancyhf{}
\fancyhead[L]{Aulas de Física}
\fancyhead[R]{\thepage}

\hypersetup{
	colorlinks=true,
	linkcolor=blue,
	urlcolor=blue
	pdftitle={Matemática - Exercícios},
	pdfauthor={Seu Nome}
}
% Configuração de links








% ========================
% comandos
% ========================

\newcommand{\mru}{\text{Movimento Retilíneo Uniforme}}
\newcommand{\eqv}{\begin{equation} v = \frac{\Delta s}{\Delta t} \end{equation}}
\newcommand{\nota}[1]{\textbf{\textit{#1}}}
\newcommand{\highlight}[1]{\textcolor{blue}{\textbf{#1}}}

\newcommand{\TabelaQuestao}[3]{
	\begin{table}[h]
		\centering
		\renewcommand{\arraystretch}{1.3}
		\begin{tabular}{ |c|c|c| }
			\hline
			Código & Matéria & Conteúdo \\ 
			\hline
			#1 & #2 & #3 \\  
			\hline
		\end{tabular}
	\end{table}
}

% Definição do comando para a tabela de prova
\newcommand{\TabelaProva}[4]{
	\begin{table}[h]
		\centering
		\renewcommand{\arraystretch}{1.3}
		\begin{tabular}{ |c|c|c|c| }
			\hline
			Ano & Banca & Órgão & Prova \\  
			\hline
			#1 & #2 & #3 & \resizebox{10cm}{!}{#4} \\
			\hline
		\end{tabular}
	\end{table}
}




% ========================
% Teoremas
% ========================

\newtheorem{formula}{Formula}[]

% ========================
% Documento Principal
% ========================
\begin{document}
	
	\title{Aulas de Física}
	\author{Isaque}
	\date{\today}
	\maketitle
	
	\tableofcontents
	
	% ========================
	% Operações Básicas
	% ========================
	\chapter{Exercícios Operações Básicas}
	\section{Lista 01 - Exercícios Operações Básicas}


\begin{enumerate}
	\item Semana 27/01/2025 - Resolva as operações a seguir e decomponha os resultados.
	
	\begin{enumerate}
		\item Adição
			\begin{enumerate}
				\item  $ 103 + 069 $
				\item  $ 1394+78998 $
				\item  $ 3,0091+267,02001$
				\item  $ 5,25 + 1,563 $
				
			\end{enumerate}
		\item Subtração
		
			\begin{enumerate}
				\item  $ 371 - 268 $
				\item  $ 26 - 123 $
				\item  $ 1,25 - 1,5$
				\item  $ 0,029 - 0,23 $
				
			\end{enumerate}
		\item Multiplicação ($ \cdot =$ x)
		
			\begin{enumerate}
				\item  $ 23 \cdot 12 $
				\item  $ 322 \cdot 100 $
				\item  $ 0,5 \cdot 3 $
				\item  $ 5,32 \cdot 23,5 $
				\item  $ 22 \cdot 33 $
				\item  $ 54 \cdot 763 $				
				\item  $ 83 \cdot 96 $
			\end{enumerate}
		
		\item Divisão
			\begin{enumerate}
				\item  $ 32 \ : 2 $
				\item  $ 45\ : 3 $
				\item  $ 125\ : 5 $
				\item  $ 520\ : 10 $
				\item  $ 3,5\ : 1,5 $
			\end{enumerate}
	
			
	\end{enumerate}
	\item Frações e porcentagem
		
		\begin{enumerate}
			\item  $ \dfrac{1}{2} + \dfrac{2}{3} $
			\item  $    \dfrac{3}{5} + \dfrac{1}{2} $
			\item  $ \dfrac{2}{4} - \dfrac{3}{2} $
			\item  $  \dfrac{1}{2} \cdot  \dfrac{3}{6} $
			\item  $  10\% \quad de \quad  1100 $
			\item  $  25\% \quad de \quad 2300 $
			\item  $  50\% \quad de \quad 3250 $
			\item  $  12\% \quad de \quad 150 $
			\item  $  15\% \quad de \quad 300 $
		\end{enumerate}
		
	\item 03/02/2025 - SEGUNDA -- Resolva
	\begin{enumerate}
		\item  $423 + 867 $
		\item  $   138 + 794 $
		\item  $   8902 + 1459 $
		\item  $   3,42 + 1,02 $
		\item  $   0,23 + 10,3 $
		\item  $   1,002 + 32,02001 $
		\item  $   723 - 541 $
		\item  $   239 + 181 $
		\item  $   -325 + (057)$
		\item  $   56 + (-99) $
		\item  $   71 - (-49) $
		
	\end{enumerate}
	

		\[
		\begin{array}{r}
			\phantom{1} \phantom{1} \phantom{1} \phantom{1}^1 \\
			\phantom{+}138 \\
			+ \phantom{0}794 \\
			\hline
		\end{array}
		\]
		
		Passo 1: Somamos as unidades (\(8 + 4 = 12\)), colocamos \(2\) e levamos \(1\):
		
		\[
		\begin{array}{r}
			\phantom{1} \phantom{1}^1 \\
			\phantom{+}138 \\
			+ \phantom{0}794 \\
			\hline
			\phantom{0} \phantom{0}2
		\end{array}
		\]
		
		Passo 2: Somamos as dezenas (\(3 + 9 = 12\)), mais \(1\) que levamos (\(12 + 1 = 13\)), colocamos \(3\) e levamos \(1\):
		
		\[
		\begin{array}{r}
			\phantom{1}^1 \phantom{1}^1 \\
			\phantom{+}138 \\
			+ \phantom{0}794 \\
			\hline
			\phantom{0} 3 2
		\end{array}
		\]
		
		Passo 3: Somamos as centenas (\(1 + 7 = 8\)), mais \(1\) que levamos (\(8 + 1 = 9\)), resultando em \(9\):
		
		\[
		\begin{array}{r}
			\phantom{1}^1 \phantom{1}^1 \\
			\phantom{+}138 \\
			+ \phantom{0}794 \\
			\hline
			9 3 2
		\end{array}
		\]
		
		O resultado final é **932**.

	
	\textbf{Subtração: \(723 - 541\)}
	
	\[
	\begin{array}{r}
		\phantom{1} \phantom{1} \phantom{1} \\
		\phantom{-}723 \\
		- \phantom{0}541 \\
		\hline
	\end{array}
	\]
	
	Passo 1: Subtraímos as unidades (\(3 - 1 = 2\)):
	
	\[
	\begin{array}{r}
		\phantom{1} \phantom{1} \phantom{1} \\
		\phantom{-}723 \\
		- \phantom{0}541 \\
		\hline
		\phantom{0} \phantom{0}2
	\end{array}
	\]
	
	Passo 2: Subtraímos as dezenas (\(2 - 4\)), como \(2\) é menor que \(4\), precisamos pegar um empréstimo da casa das centenas. O \(7\) vira \(6\) e o \(2\) vira \(12\). Agora fazemos \(12 - 4 = 8\):
	
	\[
	\begin{array}{r}
		\phantom{1}^6 \phantom{1}^{12} \\
		\phantom{-}723 \\
		- \phantom{0}541 \\
		\hline
		\phantom{0} 8 2
	\end{array}
	\]
	
	Passo 3: Subtraímos as centenas (\(6 - 5 = 1\)):
	
	\[
	\begin{array}{r}
		\phantom{1}^6 \phantom{1}^{12} \\
		\phantom{-}723 \\
		- \phantom{0}541 \\
		\hline
		1 8 2
	\end{array}
	\]
	
	O resultado final é **182**.
	
	\item 05/02/2025 - Operações 7º e 8º anos
	
	\begin{enumerate}
		\item Adição
			\begin{enumerate}
				\item $103 + 069$
				\item $268 + 35,5$
				\item $7,002 + 41,01$
				\item $-35+(-10)$
				\item $17,2+(-15,01)$
			\end{enumerate}
		\item Subtração
			\begin{enumerate}
				\item $42 - 11$
				\item $63 - 59$
				\item $17 - 23$
				\item $53,02-19,13$
				\item $26-(-8)$
				\item $-89-(+35)$
				\item $-74-(-25)$
			\end{enumerate}
		\item Exercícios extras
		\begin{enumerate}
			\item $33+45$
			\item $46+17$
			\item $7,05 + 7,2$
			\item $12,098 + 225,196$
			\item $18 + (-15)$
			\item $-26 + (-7)$
			\item $75 - 56$
			\item $38,02 - 45,189$
			\item $-5,82-0,259$
			\item $-7,86-(-10,001)$
			\item $353,113-(-0,999)$
			\item $78,634-58,5468$
		\end{enumerate}
			
		
	\end{enumerate}
	\item 06/02 - 8 ano B - Multiplicação, passei para o 6 ano também 
	\begin{enumerate}
		\item $23 \cdot 3$
		\item $245 \cdot 6$
		\item $52\cdot34$
		\item $78\cdot35$
		\item $2,72\cdot3$
		\item $97,2\cdot4$
		\item $1,728\cdot3,2$		
		\item $103,79\cdot42,3001$
		
	\end{enumerate}
	
	\item 06/02 - 8 ano B - Divisão, passei para o 6 ano também 
	\begin{enumerate}
		\item $348 \ : 2$
		\item $59\ : 3$
		\item $1038 \ : 6$
		\item $125\ : 5$
		\item $783\ : 10$
		\item $43,28\ : 2$
		\item $76,002\ : 3$
		\item $12,5\ : 2,5$
		\item $3,5 \ : 1,5$
		
	\end{enumerate}
	
\end{enumerate}



	
	
		
	% ========================
	% Intergraus Matemática
	% ========================
	\chapter{Exercícios Intergraus Matemática }
	\section{Lista 01 - Intergraus Matemática}
\url{https://www.qconcursos.com/questoes-do-enem/questoes?difficulty%5B%5D=2&discipline_ids%5B%5D=13&subject_ids%5B%5D=20321}

\begin{enumerate}
	
\begin{table}[h]
	\centering
	\renewcommand{\arraystretch}{1.3} % Espaçamento entre linhas da tabela
	\begin{tabular}{ |c|c|c| }
		\hline
		Código & Matéria & Conteúdo \\ 
		\hline
		Q2543137 & Matemática & Aritmética e Problemas, Porcentagem \\  
		\hline
	\end{tabular}
\end{table}

\begin{table}[h]
	\centering
	\renewcommand{\arraystretch}{1.3}
	\begin{tabular}{ |c|c|c|c| }
		\hline
		Ano & Banca & Órgão & Prova \\  
		\hline
		2023 & INEP & ENEM & \resizebox{10cm}{!}{INEP - 2023 - ENEM - Exame Nacional do Ensino Médio - Primeiro e Segundo - PPL (2° Aplicação)} \\
		\hline
	\end{tabular}
\end{table}
	\item Uma escola realizou uma pesquisa entre todos os seus estudantes e constatou que três em cada dez deles estão matriculados em algum curso extracurricular de língua estrangeira. \\
	Em relação ao número total de estudantes dessa escola, qual porcentagem representa o número de alunos matriculados em algum curso extracurricular de língua estrangeira?
	
	\begin{enumerate}
		\item 0,3%
		\item 0,33%
		\item 3%
		\item 30%
		\item 33%
	\end{enumerate}
 
	\newpage
	\TabelaQuestao{Q2543136}{Matemática}{Aritmética e Problemas ,
		Porcentagem} \\
	
	\TabelaProva{2023}{INEP}{ENEM}{INEP - 2023 - ENEM - Exame Nacional do Ensino Médio - Primeiro e Segundo - PPL ( 2° Aplicação)}
	
	\item Toda a iluminação de um escritório é feita utilizando-se 40 lâmpadas incandescentes que produzem 600 lúmens (lúmen = unidade de energia luminosa) cada. O gerente planeja reestruturar o sistema de iluminação desse escritório, utilizando somente lâmpadas fluorescentes que produzem 1 600 lúmens, para aumentar a quantidade de energia luminosa em 50\%. Para alcançar seu objetivo, a quantidade mínima de lâmpadas fluorescentes que o gerente desse escritório deverá instalar é
	
	\begin{enumerate}
		\item 10.
		\item 14.
		\item 15.
		\item 16.
		\item 23.		
	\end{enumerate}
	
		\newpage
	\TabelaQuestao{Q3160688}{Matemática}{Aritmética e Problemas ,
		Regra de Três} \\
	
	\TabelaProva{2024}{INEP}{ENEM}{INEP - 2024 - ENEM - Exame Nacional do Ensino Médio - PPL - primeiro e segundo dia}
	
	\item Um engenheiro civil organizou duas equipes de pedreiros, I e II, para a construção de um muro bastante extenso. Todos os pedreiros, de ambas as equipes, apresentam o mesmo rendimento por hora trabalhada frente à construção planejada. A equipe I era composta por 3 pedreiros, que construíram 36 metros quadrados de muro, em 2 dias, trabalhando 6 horas por dia. A equipe II era composta por 5 pedreiros, que trabalharam 8 horas diárias, durante 6 dias. 	Quantos metros quadrados a equipe II construiu a mais do que a equipe I?
	
	\begin{enumerate}
		\item 144
		\item 180
		\item 204
		\item 240
		\item 276
	\end{enumerate}

\end{enumerate}



	
	
	
\end{document}
