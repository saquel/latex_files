\documentclass{amsart}

\usepackage{amsmath}
\usepackage{amsfonts}

\newlength\labelwd
\settowidth\labelwd{\bfseries viii.)}

\usepackage{tasks}

\settasks{counter-format =tsk[a].), label-format=\bfseries, label-offset=1em, label-align=right, label-width
	=\labelwd, before-skip =\smallskipamount, after-item-skip=0pt}
	
\usepackage{enumitem}

\setlist[enumerate,1]{% (
	leftmargin=*, itemsep=12pt, label={\textbf{\arabic*.)}}}


\begin{document}
	
	
	\begin{center}\Large{\textbf{Elementary Number Theory}}\end{center}\vskip0.25in
	
	\begin{enumerate}[itemsep=\baselineskip]
		\item How many positive integers less than 100 have a remainder of 3 upon division by 7?
		\begin{tasks}(3)
			\task 10
			\task 11
			\task 12
			\task 13
			\task 14
		\end{tasks}
	\end{enumerate}
	
	\begin{enumerate}[start=2, itemsep=\baselineskip]
		\item For every natural number $n$, $\tau(n)$ is the number of positive divisors of $n$. Evaluate $\tau^{3}(12)$.
		\begin{tasks}(3)
			\task 1
			\task 2
			\task 3
			\task 4
			\task 6
		\end{tasks}
	\end{enumerate}
	
	
	
	\noindent {\textbf{3.) }}$p$ and $q$ are prime numbers greater than 2. which of the following statements must be true? \\
	\hspace*{3em} \hphantom{3.)\ }
	\begin{tabular}{r l}
		{\bf I}     &   \hspace*{-0.5em}$p + q$ is even. \\
		{\bf II}    &   \hspace*{-0.5em}$pq$ is odd. \\
		{\bf III}   &   \hspace*{-0.5em}$p^{2} - q^{2}$ is even
	\end{tabular}
	\begin{tabbing}
		\hspace*{2em} \= \hspace{2.5in} \= \kill
		\> {\textbf{a.) }}I only        \> {\textbf{b.) }}II only \\
		\> {\textbf{c.) }}I and II only \> {\textbf{d.) }}I and III only \\
		\> {\textbf{e.) }}I, II, and III
	\end{tabbing}
	\vskip0.25in
	
	
	\begin{enumerate}[start=4, itemsep=\baselineskip]
		\item How many integers less than 1000 are such that the remainder upon division by each of 2, 3, 4, 5, 6, and 7 is 1?
		\begin{tasks}(3)
			\task 0
			\task 1
			\task 2
			\task 3
			\task 4
		\end{tasks}
	\end{enumerate}
	
	
	\noindent {\textbf{5.) }}$n$ is a positive integer. Which of the following quantities is divisible by 3? \\
	\hspace*{3em} \hphantom{3.)\ }
	\begin{tabular}{r l}
		{\bf I}     &   \hspace*{-0.5em}$n^{3} - 1$ \\
		{\bf II}    &   \hspace*{-0.5em}$n^{3} + 1$ \\
		{\bf III}   &   \hspace*{-0.5em}$n^{3} + 2n$
	\end{tabular}
	\begin{tabbing}
		\hspace*{2em} \= \hspace{2.5in} \= \kill
		\> {\textbf{a.) }}I only        \> {\textbf{b.) }}II only \\
		\> {\textbf{c.) }}I and II only \> {\textbf{d.) }}II and III only \\
		\> {\textbf{e.) }}I, II, and III
	\end{tabbing}
		
	
\end{document}