\documentclass[11pt,a4paper]{article}

\usepackage[portuguese]{babel}
\usepackage[utf8]{inputenc}
\usepackage{setspace}
\usepackage[hmargin={2cm}]{geometry}
\usepackage{graphicx}
\usepackage{indentfirst}

\begin{document}
	Entre os colchetes do \itshape \textbf{documentclass} \normalfont posso utilizar algumas opções, entre elas são:
	
\begin{enumerate}
	\item dsadsa
\end{enumerate}	
	
\Large Durante o texto podemos utilizar algumas \itshape \textbf {formatações de fontes} \normalfont das quais listo logo abaixo:\\ \normalsize

\begin{enumerate}
	\item Font Styles
		\begin{enumerate}
			\item {\LARGE{\textit{Texto em Itálico}}}
				\begin{enumerate}
					\item o \textbackslash textit é igual o \textbackslash itshape
				\end{enumerate}
			\item {\LARGE\textbf{Texto em Negrito}}
				\begin{enumerate}
					\item o \textbackslash textbf é igual o \textbackslash bfseries
				\end{enumerate}
			\item {\LARGE\textsc{Texto em Pequenas letras maiúsculas}}
				\begin{enumerate}
					\item o \textbackslash textsc é igual o \textbackslash scshape
				\end{enumerate}
			\item {\normalfont\LARGE{Normal fonte}}
			\begin{enumerate}
				\item o \textbackslash normalfont é igual o \textbackslash scshape
			\end{enumerate}	
				
		\end{enumerate}
	\item Default font families
		\begin{enumerate}
			\item {\LARGE{\texttt{typewriter (monospace)}}}
			\begin{enumerate}
				\item o \textbackslash texttt é igual o \textbackslash ttfamily
			\end{enumerate}
			\item {\LARGE\textsf{sans serif}}
			\begin{enumerate}
				\item o \textbackslash textsf é igual o \textbackslash sffamily
			\end{enumerate}
			\item {\LARGE\textrm{serif (roman)}}
			\begin{enumerate}
				\item o \textbackslash textrm é igual o \textbackslash rmfamily
			\end{enumerate}
		\end{enumerate}
	\item Font sizes
		\begin{itemize}
			\item{
			\verb=\tiny         =  {\ttfamily \tiny  5pt 	}\\
			\verb=\scriptsize   =  {\ttfamily \scriptsize 7pt 	}\\
			\verb=\footnotesize =  {\ttfamily \footnotesize 8pt		}\\
			\verb=\small        =  {\ttfamily \small 9pt 	}\\
			\verb=\normalsize   =  {\ttfamily \normalsize 10pt	}\\
			\verb=\large        =  {\ttfamily \large 12pt	}\\
			\verb=\Large        =  {\ttfamily \Large 14.4pt 	}\\ 
			\verb=\LARGE        =  {\ttfamily \LARGE 17.28pt	}\\
			\verb=\huge         =  {\ttfamily \huge 20.74pt }\\
			\verb=\Huge         =  {\ttfamily \Huge 24.88pt }\\
			}
		\end{itemize}
	\item Espaçamentos
		\begin{enumerate}
			\item Entre palavras
				\begin{itemize}
					\item \textbackslash quad 
					
				\end{itemize}
			\item Entre linhas
				\begin{itemize}
					\item Para utilizar o espaçamento entre linhas utilizamos o pacote \textbackslash usepackage\{setspace\}. \\ Esses comandos terão efeito até à proxima instrução que altere o espaçamento.
					\item \textbackslash{}singlespacing Para um espaçamento simples
					\item \textbackslash{}onehalfspacing Para um espaçamento de 1,5
					\item \textbackslash{}doublespacing Para um espaçamento duplo
				
				\end{itemize}
		\end{enumerate}
\end{enumerate}

\Large \textbf{Outros tipos de caracter que podem ser uteis}\\ \normalsize
	\begin{enumerate}
		\item Barra invertida (Backslash)
		\begin{itemize}
			\item Para utiliar a barra invertida deve utilizar o comando \textbackslash textbackslash
			\item Ou podemos utilizar dentro de uma formula matemática \$$\backslash$backslash\$
		\end{itemize}
		\item Arrows (Setas)
		\begin{itemize}
			\item  $\backslash$ \textbackslash longleftarrow \quad $\longleftarrow$ 
			\item $\backslash$ longleftrightarrow \quad $\longleftrightarrow$
			\item $\backslash$ longmapsto \quad $\longmapsto$
			\item $\backslash$ longrightarrow \quad$\longrightarrow$
			\item $\backslash$ arrowvert \quad $\arrowvert$
		\end{itemize}
	
\end{enumerate}
	
\newpage

\begin{minipage}[c][1.5cm][c]{3.5cm}
\includegraphics[height=1cm]{imagens/logo.png}
\end{minipage}
\begin{minipage}[c][1.5cm][c]{10cm}
	 \centering \textsc{Ministério da Educação}
\end{minipage}
\begin{minipage}[c][1.5cm][c]{5cm}
	Data: 16/dez/2024
	
	Ano: 8º
	
	Turma: B
\end{minipage}

\vspace{.5cm}

Nome:\hrulefill \, N.o: \rule{.5cm}{.1mm}
\vspace{1cm}

Classificação: \hrulefill \, O Professor: \hrulefill


\end{document}